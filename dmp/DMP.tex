% Options for packages loaded elsewhere
\PassOptionsToPackage{unicode}{hyperref}
\PassOptionsToPackage{hyphens}{url}
%
\documentclass[
]{article}
\usepackage{lmodern}
\usepackage{amssymb,amsmath}
\usepackage{ifxetex,ifluatex}
\usepackage[margin=1.4cm]{geometry}

\ifnum 0\ifxetex 1\fi\ifluatex 1\fi=0 % if pdftex
  \usepackage[T1]{fontenc}
  \usepackage[utf8]{inputenc}
  \usepackage{textcomp} % provide euro and other symbols
\else % if luatex or xetex
  \usepackage{unicode-math}
  \defaultfontfeatures{Scale=MatchLowercase}
  \defaultfontfeatures[\rmfamily]{Ligatures=TeX,Scale=1}
\fi
% Use upquote if available, for straight quotes in verbatim environments
\IfFileExists{upquote.sty}{\usepackage{upquote}}{}
\IfFileExists{microtype.sty}{% use microtype if available
  \usepackage[]{microtype}
  \UseMicrotypeSet[protrusion]{basicmath} % disable protrusion for tt fonts
}{}
\makeatletter
\@ifundefined{KOMAClassName}{% if non-KOMA class
  \IfFileExists{parskip.sty}{%
    \usepackage{parskip}
  }{% else
    \setlength{\parindent}{0pt}
    \setlength{\parskip}{6pt plus 2pt minus 1pt}}
}{% if KOMA class
  \KOMAoptions{parskip=half}}
\makeatother
\usepackage{xcolor}
\IfFileExists{xurl.sty}{\usepackage{xurl}}{} % add URL line breaks if available
\IfFileExists{bookmark.sty}{\usepackage{bookmark}}{\usepackage{hyperref}}
\hypersetup{
  pdftitle={Data Management Plan},
  hidelinks,
  pdfcreator={LaTeX via pandoc}}
\urlstyle{same} % disable monospaced font for URLs
\setlength{\emergencystretch}{3em} % prevent overfull lines
\providecommand{\tightlist}{%
  \setlength{\itemsep}{0pt}\setlength{\parskip}{0pt}}
\setcounter{secnumdepth}{-\maxdimen} % remove section numbering

\title{Predicting influenza occurrences based on weather}
\author{}
\date{}

\begin{document}
\maketitle

\hypertarget{fair-data-science}{%
\section{FAIR Data Science}\label{fair-data-science}}

\begin{description}
\item[Contact person]
\begin{itemize}
\tightlist
\item
  {Ondrej Hudcovic}
  ({\href{mailto:e11919907@student.tuwien.ac.at}{\nolinkurl{e11919907@student.tuwien.ac.at}}})
\end{itemize}
\item[Based on]
Common DSW Knowledge Model, 2.0.1 ({{dsw}:{root}:{2.0.1}})
\item[Generated on]
20 Apr 2020
\end{description}

Data Management Plan created in Data Stewardship Wizard
\textless{}\url{https://ds-wizard.org}\textgreater{}

\hypertarget{abstract}{%
\subsection{Abstract}\label{abstract}}

The goal of this experiment was to model the relationship between
weather observations and the prevalence of new influenza infections. It
included reading, preparing and transforming data. Subsequently, these
data were visualized and used for building a prediction model - we tried
to predict influenza infections based on weather conditions.

\hypertarget{dmp-content}{}
\hypertarget{sec-data-collection}{}
\hypertarget{section-a-data-collection}{%
\subsection{Section A: Data
Collection}\label{section-a-data-collection}}

\hypertarget{q-what-data}{}
\hypertarget{what-data-will-you-collect-or-create}{%
\subsubsection{1. What data will you collect or
create?}\label{what-data-will-you-collect-or-create}}

\hypertarget{instrument-datasets}{%
\paragraph{Instrument datasets}\label{instrument-datasets}}

The following instrument datasets will be acquired in the project:

\begin{itemize}
\item \textbf{\href{https://www.data.gv.at/katalog/dataset/grippemeldedienst-stadt-wien}{influenza.csv}} - number of influenza occurences in Vienna (weekly data, 2009 - 2018)
  \item \textbf{\href{https://www.meteoblue.com/en/weather/archive/export/vienna_austria_2761369}{weather observations}} - temperature, humidity, wind and similar data for Vienna (2012-2018)


  This dataset will be collected by experts in the project, with our own
  equipment.

  The equipment is very well described and known.
\end{itemize}

\hypertarget{data-formats-and-types}{%
\paragraph{Data formats and types}\label{data-formats-and-types}}

We will be using the following data formats and types:

\begin{itemize}
\item
  \textbf{\href{https://fairsharing.org/bsg-s001303}{CSV Dialect
  Description Format}}

  It is a standardized format. This is a suitable format for long-term
  archiving. We will have only a small amount of data stored in this
  format.
\end{itemize}

\hypertarget{q-how-data}{}
\hypertarget{how-will-the-data-be-collected-or-created}{%
\subsubsection{2. How will the data be collected or
created?}\label{how-will-the-data-be-collected-or-created}}

\hypertarget{instrument-datasets-1}{%
\paragraph{Instrument datasets}\label{instrument-datasets-1}}

\begin{itemize}
  \item \textbf{\href{https://www.data.gv.at/katalog/dataset/grippemeldedienst-stadt-wien}{influenza.csv}} - number of influenza occurences in Vienna (weekly data, 2009 - 2018)
  \item \textbf{\href{https://www.meteoblue.com/en/weather/archive/export/vienna_austria_2761369}{weather observations}} - temperature, humidity, wind and similar data for Vienna (2012-2018)

  No instruments for this dataset have been specified.

  We will not be using quality process for this dataset.
\end{itemize}

\hypertarget{storage-and-file-conventions}{%
\paragraph{Storage and file
conventions}\label{storage-and-file-conventions}}

We will use a filesystem with files and folders with the following
folder conventions:

\begin{itemize}
\tightlist
\item
  There will be a \textbf{folder for each sample/subject}. Each of those
  will use the following conventions:
  \begin{itemize}
      \item \textbf{data} - these are the datasets used in the experiment
      \item \textbf{images} - generated images from "visualization" part of the project
      \item \textbf{src} - source code
  \end{itemize} 
\end{itemize}
Every "type" of data (be it source code, datasets, ...) has its own designated folder in the repository.

Moreover, we have made appointments about naming the files.

We will not be storing data in an "object store" system.

We will not use a relational database system to store project data.

We will not use a graph database for data in the project.

We will not be storing data in a triple store.

\hypertarget{sec-docs-metadata}{}
\hypertarget{section-b-documentation-and-meta-data}{%
\subsection{Section B: Documentation and
Meta-data}\label{section-b-documentation-and-meta-data}}

\hypertarget{q-docs-metadata}{}
\hypertarget{what-documentation-and-meta-data-will-accompany-the-data}{%
\subsubsection{3. What documentation and meta-data will accompany the
data?}\label{what-documentation-and-meta-data-will-accompany-the-data}}

List of data to be published is given in Section E, Question 9. This
also includes information about catalogs where the data can be found.
Information about data types used is given in Section A, Question 1.

We will use an electronic lab notebook to make sure that there is good
provenance of the data analysis.

We will be documenting the data with W3C PROV provenance.

\hypertarget{sec-ethics-legal}{}
\hypertarget{section-c-ethics-and-legal-compliance}{%
\subsection{Section C: Ethics and Legal
Compliance}\label{section-c-ethics-and-legal-compliance}}

\hypertarget{q-ethical-issues}{}
\hypertarget{how-will-you-manage-any-ethical-issues}{%
\subsubsection{4. How will you manage any ethical
issues?}\label{how-will-you-manage-any-ethical-issues}}

\hypertarget{q-ethical-issues}{}
\hypertarget{how-will-you-manage-copyright-and-intellectual-property-rights-ipr-issues}{%
\subsubsection{5. How will you manage copyright and Intellectual
Property Rights (IPR)
issues?}\label{how-will-you-manage-copyright-and-intellectual-property-rights-ipr-issues}}

We will be working with the philosophy \emph{as open as possible} for
our data.

All of our data can become completely open immediately.

\hypertarget{sec-storage-backup}{}
\hypertarget{section-d-storage-and-backup}{%
\subsection{Section D: Storage and
Backup}\label{section-d-storage-and-backup}}

\hypertarget{q-store-backup}{}
\hypertarget{how-will-the-data-be-stored-and-backed-up-during-the-research}{%
\subsubsection{6. How will the data be stored and backed up during the
research?}\label{how-will-the-data-be-stored-and-backed-up-during-the-research}}

Storage needs will be the same during the whole project.

All essential data is also stored elsewhere to prevent a total loss of
data. All project data stored outside of the working area will be
adequately backed up.

\hypertarget{q-access-security}{}
\hypertarget{how-will-you-manage-access-and-security}{%
\subsubsection{7. How will you manage access and
security?}\label{how-will-you-manage-access-and-security}}

Project members will not store data or software on computers in the lab
or external hard drives connected to those computers.They will not carry
data with them (e.g. on laptops, USB sticks, or other external media).
All data centers where project data is stored carry sufficient
certifications. All project web services addressed via secure http
(https://...). Project members have been instructed about both generic
and specific risks to the project.

The possible impact to the project or organization if information is
lost is small. The possible impact to the project or organization if
information is leaked is small. The possible impact to the project or
organization if information is vandalised is small.

We are not using any personal information.

Only project members will have read access; only selected project
members will be able to write data.

\hypertarget{sec-selection-preservation}{}
\hypertarget{section-e-selection-and-preservation}{%
\subsection{Section E: Selection and
Preservation}\label{section-e-selection-and-preservation}}

\hypertarget{q-which-longterm}{}
\hypertarget{which-data-are-of-long-term-value-and-should-be-retained-shared-andor-preserved}{%
\subsubsection{8. Which data are of long-term value and should be
retained, shared, and/or
preserved?}\label{which-data-are-of-long-term-value-and-should-be-retained-shared-andor-preserved}}

We plan to publish the following datasets:

\begin{itemize}
\tightlist
\item
  \textbf{\href{https://www.data.gv.at/katalog/dataset/grippemeldedienst-stadt-wien}{influenza.csv}, \href{https://www.meteoblue.com/en/weather/archive/export/vienna_austria_2761369}{weather observations}} {--} This data set will
  be kept available as long as technically possible. {--} The metadata
  will be available even when the data no longer exists.
\end{itemize}

\hypertarget{q-longterm-plan}{}
\hypertarget{what-is-the-longterm-preservation-plan-for-the-dataset}{%
\subsubsection{9. What is the longterm preservation plan for the
dataset?}\label{what-is-the-longterm-preservation-plan-for-the-dataset}}

\begin{itemize}
\tightlist
\item
  \textbf{\href{https://www.data.gv.at/katalog/dataset/grippemeldedienst-stadt-wien}{influenza.csv}, \href{https://www.meteoblue.com/en/weather/archive/export/vienna_austria_2761369}{weather observations}} will be stored in a
  domain-specific repository:
  \href{https://fairsharing.org/bsg-d001160}{GitHub}. \href{https://github.com/hudcondr/data-stewardship-ex1}{This is the link to the repository of the project}. We don't need to
  contact the repository because it is a routine for us. We will be
  adding a reference to the published data to at least one data
  catalogue.
\end{itemize}

None of the used repositories charge for their services.

We have a reserved budget for the time and effort it will take to
prepare the data for publication.

\hypertarget{sec-data-sharing}{}
\hypertarget{section-f-data-sharing}{%
\subsection{Section F: Data Sharing}\label{section-f-data-sharing}}

\hypertarget{q-how-share}{}
\hypertarget{how-will-you-share-the-data}{%
\subsubsection{10. How will you share the
data?}\label{how-will-you-share-the-data}}

\begin{itemize}
\tightlist
\item
  \textbf{\href{https://www.data.gv.at/katalog/dataset/grippemeldedienst-stadt-wien}{influenza.csv}, \href{https://www.meteoblue.com/en/weather/archive/export/vienna_austria_2761369}{weather observations}} {--} freely available for
  any use (public domain or CC0).
\end{itemize}

Information about used repositories (i.e. where will potential users
find out about the data) is provided in Section E, Question 9.

Embargo on the data is described in Section C, Question 5, and Section
F, Question 11.

\hypertarget{q-restrictions}{}
\hypertarget{are-any-restrictions-on-data-sharing-required}{%
\subsubsection{11. Are any restrictions on data sharing
required?}\label{are-any-restrictions-on-data-sharing-required}}

Ethical and legal restrictions are documented under Section C. We have
used the Data Stewardship Wizard, which made us aware of options to
minimize the restrictions.

No data sharing agreement will be required.

\hypertarget{sec-responsibilities-resources}{}
\hypertarget{section-g-responsibilities-and-resources}{%
\subsection{Section G: Responsibilities and
Resources}\label{section-g-responsibilities-and-resources}}

\hypertarget{q-dm-responsible}{}
\hypertarget{who-will-be-responsible-for-data-management}{%
\subsubsection{12. Who will be responsible for data
management?}\label{who-will-be-responsible-for-data-management}}

Ondrej Hudcovic is responsible for implementing the DMP, and ensuring it
is reviewed and revised.

\hypertarget{q-required-resources}{}
\hypertarget{what-resources-will-you-require-to-deliver-your-plan}{%
\subsubsection{13. What resources will you require to deliver your
plan?}\label{what-resources-will-you-require-to-deliver-your-plan}}

To execute the DMP, no additional specialist expertise is required.

We do not require any hardware or software in addition to what is
usually available in the institute.

Charges applied by data repositories (if any) are mentioned already in
Section E, Question 9.

\end{document}

